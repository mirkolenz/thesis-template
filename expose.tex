\documentclass{scrartcl}
\usepackage{_init}

\title{Title of Exposé}
\author{Author Name}
\date{\today}

\addbibresource{references.bib}

\DeclareAcronym{aif}{short=AIF, long=Argument Interchange Format,
cite={Chesnevar2006ArgumentInterchangeFormat}}


\begin{document}

\maketitle


\section{Introduction}\label{sec:introduction}
Short summary of your planned project.
It should include
\begin{enumerate*}
    \item a motivation for your work,
    \item potential use cases,
    \item an abstract version of your research question, and
    \item an overview of the main contributions.
\end{enumerate*}
\emph{Important:}
This exposé shall be at most six pages long (including references).


\section{Foundations}\label{sec:foundations}
Short introduction to the concepts that are necessary for the remainder of the exposé.
Our template offers two ways to cite:
\begin{itemize}
    \item Parentheses: \autocite{Mikolov2013EfficientEstimationWord}
    \item Inline: \textcite{Mikolov2013EfficientEstimationWord}
\end{itemize}


\section{Formulation of the Research Question}\label{sec:research-question}
Describe the problem you are faced with in more detail.
This section builds on the introduction, but you should go into more detail since the foundations are now known.


\section{Related Work}\label{sec:related-work}
Describe how others tackled the described problem or which works have similar goals.
This section requires a preliminary literature research.


\section{Methodology and Approach}\label{sec:methodology}
Explain \emph{how} you plan to tackle the described problem.
You should include potential approaches and methods that you plan to use.
However, do not present a finished solution---that is the goal of your thesis.
Ideally, include a few working hypotheses that will be investigated in your thesis (e.g., in your experimental evaluation).


\section{Draft Structure}\label{sec:draft}
Overview of the structure planned for your thesis.

\begin{easylist}[articletoc]
    # Introduction
    ## Motivation
    ## Contribution
    ## Structure

    # Foundations
    ## \dots
\end{easylist}


\section{Time Schedule}\label{sec:schedule}
Define tasks/milestones and their deadlines for your thesis.

\begin{longtable}{ll}
    \toprule

    \textbf{Work Package} & \textbf{Period} \\%
    \midrule%
    \endhead%

    Literature research.  & \dots           \\
    \dots                 & \dots           \\

    \bottomrule
\end{longtable}



\printbibliography


\end{document}
